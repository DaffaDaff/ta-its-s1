\chapter{PENDAHULUAN}

\section{Latar Belakang}

Penyu merupakan salah satu spesies yang dilindungi karena populasinya terancam punah di seluruh dunia. 
Di Indonesia sendiri terdapat 6 dari 7 jenis penyu yang ada di seluruh dunia. 
Penyu telah mengalami penurunan jumlah populasi secara signifikan dalam jangka waktu terakhir ini. 
Di habitatnya, penyu-penyu yang baru menetas menghadapi berbagai ancaman kematian dari hewan-hewan seperti burung, kepiting, dan biawak. 
Namun, ancaman terbesar yang mengakibatkan penurunan populasi yang signifikan adalah manusia. 
Berbagai pembangunan di daerah pesisir yang berlebihan mengakibatkan pengurangan habitat penyu untuk bersarang. 
Penangkapan dan pemburuan penyu untuk diambil telur, daging, kulit, dan cangkangnya membuat populasi penyu di seluruh dunia berkurang. 
\parencite{JKT602}

Beberapa kasus kematian penyu yang terjadi di Sulawesi Barat akibat penangkapan ilegal diantaranya pada tahun 2016 kasus kematian 2 ekor penyu di Pantai Palippis dan Pantai Lapeo yang ditemukan dalam keadaan penyu luka dibagian dubur yang menandakan telurnya telah dikeluarkan dari tubuhnya secara paksa. 
Pada Tahun 2017, 3 ekor penyu ditemukan dalam keadaan tubuhnya tidak utuh di Pantai Mampie.  
Kemudian pada Tahun 2019 kematian penyu berjumlah 18 ekor di Pantai Mampie, kebanyakan penyu yang ditemukan mati tersebut diduga dibunuh dengan sengaja untuk diambil telur dan dagingnya. 
\parencite{nur2022pelatihan}

Konservasi merupakan salah satu cara yang diharapkan dapat mencegah punahnya populasi penyu.
Salah satu metode konservasi adalah dengan mempelajari perilaku dan pola hidup penyu dengan cara melacak penyu pada habitatnya.
Dengan mempelajari perilaku dan pola hidup penyu, peneliti dan konservasionis dapat mengetahui bagaimana cara membantu dalam pelestarian penyu, 
salah satunya adalah dengan cara membangun taman konservasi pada daerah tempat mereka bertelur serta mengawasi habitat kritikal dan jalur migrasi mereka. 
Teknologi pelacakan yang telah diimplementasikan adalah menggunakan satellite tagging.
Satellite tagging atau satellite tracking adalah metode pelacakan posisi penyu dengan memasang modul GPS dengan presisi tinggi.
Satellite tagging memungkinkan peneliti dan konservasionis untuk melacak posisi dan perlakuan penyu pada habitatnya dan melacak jalur migrasinya.
Penggunaan satellite tagging memerlukan pemasangan yang tepat untuk memastikan pemasang tidak menyakiti penyu.
Penggunaan satellite tagging juga memerlukan modul GPS yang tidak murah untuk setiap penyu.
Selain itu, penggunaan satellite tracking juga memiliki resiko malfungsi dan rusak.
\parencite{10.3389/fmars.2018.00432}

Oleh karena itu, penelitian ini ditujukan untuk mencari alternatif dalam mengidentifikasi dan melacak individu penyu pada habitatnya.
Solusi yang ditawarkan pada penelitian ini adalah re-identifikasi (Re-ID) dengan menggunakan deep learning.
Re-identifikasi adalah proses mengidentifikasi kembali suatu objek atau individu yang telah diidentifikasi sebelumnya dengan melihat fitur-fitur visual yang unik.
Re-identifikasi dapat mengidentifikasi suatu individu dengan perbedaan sudut kamera dan pencahayaan.
\parencite{zheng2016personreidentificationpastpresent}
Penggunaan teknologi deep learning, seperti Vision Transformer (ViT) dapat mengekstrak fitur pada individu dengan efektif yang memungkinkan identifikasi individu pada berbagai sudut.
ViT juga menawarkan performa dan akurasi yang lebih dibandingkan dengan model lain seperti CNN.
\parencite{9716741}
Dengan menggunakan fitur tersebut, penggunaan ViT dapat digunakan untuk mengidentifikasi pola unik yang ada pada kepala setiap individu penyu.


\section{Rumusan Masalah}

Berdasarkan latar belakang di atas didapatkan beberapa rumusan masalah.
Pertama, bagaimana cara mengidentifikasi penyu secara non-invasif menggunakan deep learning.
Kedua, bagaimana efektivitas penggunaan Vision Transformer dalam Re-identifikasi penyu.

\section{Batasan Masalah}

Penelitian ini memiliki batasan masalah untuk menetapkan fokus penelitian. Batasan masalah meliputi:
\begin{enumerate}
    \item Penelitian ini berfokus pada pengembangan model Vision Transformer yang digunakan untuk re-identifikasi penyu hijau.
    \item Dataset yang digunakan berasal dari dataset yang telah dikumpulkan oleh wildlifedatasets dan tersedia untuk pemakaian publik.
    \item Dataset yang digunakan berisi citra kepala penyu hijau yang diambil dari berbagai sudut dan pencahayaan.
\end{enumerate}

\section{Tujuan}

Tujuan dari penelitian ini adalah untuk mengembangkan model Vision Transformer yang dapat melakukan re-identifikasi pada penyu hijau berdasarkan pola di kepala dengan akurasi yang memuaskan.

\section{Manfaat}

Penelitian diharapkan memberikan manfaat berupa solusi pelacakan penyu secara non-invasif yang dapat digunakan untuk penelitian dan tindakan dalam konservasi penyu. 
Selain itu, penelitian ini juga diharapkan dapat memberikan model identifikasi penyu yang akurat.