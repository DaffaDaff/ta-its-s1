\chapter*{ABSTRACT}
\begin{center}
  \large
  \textbf{RE-IDENTIFICATION OF SEA TURTLES USING DEEP LEARNING FOR ENDANGERED ANIMAL CONSERVATION}
\end{center}
% Menyembunyikan nomor halaman
\thispagestyle{empty}

\begin{flushleft}
  \setlength{\tabcolsep}{0pt}
  \bfseries
  \begin{tabular}{lc@{\hspace{6pt}}l}
  Student Name / NRP&: &Sulthan Daffa Arif Mahmudi / 5024211005\\
  Department&: &Computer Engineering FTEIC - ITS\\
  Advisor&: &1. Reza Fuad Rachmadi, S.T., M.T., Ph.D\\
  & & 2. Prof. Dr. I Ketut Eddy Purnama, S.T., M.T\\
  \end{tabular}
  \vspace{4ex}
\end{flushleft}
\textbf{Abstract}

% Isi Abstrak
Sea turtles are a species facing the risk of extinction worldwide. 
Sea turtle populations have declined significantly due to poaching, illegal trade, and habitat destruction. 
To address this risk, conservation efforts are needed to prevent a drastic population decline. 
One method of sea turtle conservation is tagging, which aims to study the behavior and habitat of sea turtles to protect their critical habitat areas.
A commonly used tagging method is satellite tracking, which involves attaching high-precision GPS modules to monitor sea turtles in real-time. 
However, this method poses challenges, including complex installation and the potential to increase sea turtle stress due to its invasive nature. 
Additionally, satellite tagging requires substantial costs.
As an alternative, this study proposes an AI-based tagging approach through re-identification techniques. 
By using deep learning, a re-identification model can be developed by extracting unique patterns on the sea turtle's head as visual features. 
This re-identification model is trained using the Vision Transformer (ViT), and the dataset used consists of images of green sea turtle heads from various camera angles that available publicly.

\vspace{2ex}
\noindent
\textbf{Keywords: \emph{Re-Identification, Vision Transformer, Sea Turtles, Conservation}}