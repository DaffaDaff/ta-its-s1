\chapter*{ABSTRAK}
\begin{center}
  \large
  \textbf{RE-IDENTIFIKASI PENYU DENGAN MENGGUNAKAN DEEP LEARNING UNTUK KONSERVASI HEWAN LANGKA}
\end{center}
\addcontentsline{toc}{chapter}{ABSTRAK}
% Menyembunyikan nomor halaman
\thispagestyle{empty}

\begin{flushleft}
  \setlength{\tabcolsep}{0pt}
  \bfseries
  \begin{tabular}{ll@{\hspace{6pt}}l}
  Nama Mahasiswa / NRP&:& Sulthan Daffa Arif Mahmudi / 5024211005\\
  Departemen&:& Teknik Komputer FTEIC - ITS\\
  Dosen Pembimbing&:& 1. Reza Fuad Rachmadi, S.T., M.T., Ph.D\\
  & & 2. Prof. Dr. I Ketut Eddy Purnama, S.T., M.T\\
  \end{tabular}
  \vspace{4ex}
\end{flushleft}
\textbf{Abstrak}

% Isi Abstrak
Penyu merupakan spesies yang menghadapi risiko kepunahan di seluruh dunia. 
Populasi penyu mengalami penurunan signifikan akibat perburuan dan perdagangan ilegal serta kerusakan habitatnya. 
Untuk mengatasi risiko ini, diperlukan upaya konservasi guna mencegah penurunan populasi yang drastis. 
Salah satu metode konservasi penyu adalah dengan melakukan tagging, yang bertujuan untuk mempelajari pola hidup dan habitat penyu guna melindungi area habitat kritis mereka.
Metode tagging yang umum digunakan adalah pelacakan satelit, dengan memasang modul GPS presisi tinggi untuk memantau penyu secara real-time. 
Namun, metode ini memiliki tantangan, termasuk pemasangan yang kompleks dan potensi meningkatkan stres pada penyu karena sifatnya yang invasif. 
Selain itu, tagging satelit memerlukan biaya yang cukup tinggi.
Sebagai alternatif, penelitian ini mengusulkan pendekatan tagging berbasis kecerdasan buatan melalui teknik Re-Identifikasi. 
Dengan menggunakan deep learning, model re-identifikasi dapat dikembangkan dengan mengekstraksi pola unik pada kepala penyu sebagai fitur visual. 
Model Re-identifikasi ini dilatih menggunakan Vision Transformer (ViT), dan data yang digunakan adalah gambar kepala penyu hijau dari berbagai sudut kamera yang tersedia secara publik.

\vspace{2ex}
\noindent
\textbf{Kata Kunci: \emph{Re-identifikasi, Vision Transformer, Penyu, Konservasi}}